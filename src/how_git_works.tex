\documentclass{beamer}
\usepackage{graphicx}
\usepackage{hyperref}
\usepackage{appendixnumberbeamer}
\usepackage{fontawesome}

\usetheme[numbering=fraction,block=fill,progressbar=frametitle]{metropolis}
\usecolortheme{owl}

\hypersetup{
    colorlinks=false,
    pdftitle={Introduction to Git},
    pdfpagemode=FullScreen,
    pdfauthor={Jaroslaw Karczmarczyk}
}

\urlstyle{same}
\graphicspath{ {../images/} }

\setbeameroption{show notes}
\setbeamercovered{transparent}
\setbeamertemplate{note page}[compressed]
\setbeamerfont{note page}{size=\tiny}
\addtobeamertemplate{note page}{\setbeamerfont{itemize/enumerate subbody}{size=\tiny}}{}

\title{Introduction to \includegraphics[height=1.2ex]{git-logo-white}}
\subtitle{How \includegraphics[height=1.2ex]{git-logo-white} works}
\date{\today}
\author[Jarek]{Jarosław Karczmarczyk\\\texttt{jaroslaw.karczmarczyk@pwc.com}}
\institute{PwC Advisory spółka z ograniczoną odpowiedzialnością sp.k.\\Data \& Analytics}
\logo{\includegraphics[scale=0.08]{pwc-logo-white}\hspace{6pt}}
\titlegraphic{\hfill\includegraphics[scale=0.12]{pwc-logo-white}}
\setbeamercolor{alerted text}{fg=OwlYellow}
\setbeamercolor*{block title}{
    use=normal text,
    fg=normal text.fg,
    bg=normal text.bg!80!fg
}
\setbeamercolor{block body}{
    use={block title, normal text},
    bg=block title.bg!50!normal text.bg
}

\setbeamertemplate{itemize items}[triangle]

\begin{document}
    \maketitle

    \begin{frame}{Introduction}
        Porcelain commands --- high level commands
        \begin{itemize}[<+-| alert@+>]
            \item\texttt{git status}
            \item\texttt{git add}
            \item\texttt{git commit}
            \item\texttt{git branch}
            \item\texttt{git switch} (\texttt{git checkout})
            \item\texttt{git pull}
            \item\texttt{...}
        \end{itemize}
        %\note{aa}
    \end{frame}

    \begin{frame}{Introduction}
        Plumbing commands --- low level commands
        \begin{itemize}[<+-| alert@+>]
            \item\texttt{git cat-file}
            \item\texttt{git hash-object}
            \item\texttt{git count-objects}
            \item\texttt{...}
        \end{itemize}
    \end{frame}

    \begin{frame}{Introduction}
        \begin{center}
            \onslide<1->{If you want to master \alert{Git}, don't worry\\about learning the commands.}

            \onslide<2->{Instead, learn the model.}
        \end{center}
    \end{frame}

    \begin{frame}{What is Git?}
        \begin{center}
            \alert{Git} is a Distributed Version Control System.
        \end{center}
        \pause
        \begin{center}
            That’s a lot of words to define \alert{Git}.
        \end{center}
    \end{frame}

    \begin{frame}{What is Git?}
        \begin{center}
            \only<1>{\alert{Git} is a Version Control System.}
            \only<2>{\alert{Git} is a Stupid Content Tracker.}
            \only<3>{\alert{Git} is a Persistent Map.}
        \end{center}
    \end{frame}

    \begin{frame}{Values and keys}
        \begin{center}
            \only<1>{\alert{Value} is a sequence of bytes.}
            \only<2>{\alert{Key} is a \alert{SHA1}\footnote{\url{https://en.wikipedia.org/wiki/SHA-1}} hash.}
        \end{center}
    \end{frame}

    \begin{frame}{Values and keys}
        \begin{center}
            \begin{tabular}{@{}l@{}@{}l l@{}}
                value: & & \texttt{test} \\
                key:   & & \texttt{9daeafb9864cf43055ae93beb0afd6c7d144bfa4}
            \end{tabular}
            \\[1em]
            \pause
            \texttt{echo "test" | git hash-object --stdin}
        \end{center}
    \end{frame}

    \begin{frame}{Values and keys}
        \begin{center}
            Every object in \alert{Git} has its own SHA1 hash.
        \end{center}
        \pause
        \begin{center}
            So, what if they collide?
        \end{center}
    \end{frame}

    \begin{frame}{A persistent map}
        \begin{center}
            \only<1>{\texttt{echo "test" | git hash-object --stdin -w}}
            \only<2>{\texttt{git cat-file -t 9daeafb} (type)}
            \only<3>{\texttt{git cat-file -p 9daeafb} (pretty print)}
        \end{center}
    \end{frame}

    \begin{frame}{Example}
        \begin{center}
            First commit!
        \end{center}
        \note[item]{Przejdz do katalogu \texttt{\textasciitilde/projects/training}}
        \note[item]{\texttt{tree .}}
        \note[item]{\texttt{git init}}
        \note[item]{\texttt{l} --- pojawił się katalog \texttt{.git}}
        \note[item]{katalog \texttt{.git/objects} jest pusty}
        \note[item]{\texttt{git status}}
        \note[item]{\texttt{git add main\_file.txt}}
        \note[item]{\texttt{git add config}}
        \note[item]{\texttt{git status}}
        \note[item]{\texttt{git commit -m "1st commit"}}
        \note[item]{\texttt{git status}}
        \note[item]{\texttt{git log}}
        \note[item]{\texttt{open .git}}
        \note[item]{Pokaż za pomocą \texttt{git cat-file -p} zawartość commita, tree i blobs}
        \note[item]{Zmodyfikuj \texttt{main\_file.txt}}
        \note[item]{\texttt{git status}}
        \note[item]{\texttt{git add main\_file.txt}}
        \note[item]{\texttt{git commit -m "2nd commit"}}
        \note[item]{\texttt{git status}}
        \note[item]{\texttt{git log}}
        \note[item]{Pokaż za pomocą \texttt{git cat-file -p} zawartość commita --- parent}
        \note[item]{Pożaż, jak zmieniło się tree}
        \note[item]{\texttt{git count-objects}}
    \end{frame}

    \begin{frame}{Git objects}
        \begin{center}
            What about large files?
        \end{center}
    \end{frame}

    \begin{frame}{Git objects}
        \begin{itemize}[<+-| alert@+>]
            \item Blobs
            \item Trees
            \item Commits
            \item Annotated Tags
        \end{itemize}
    \end{frame}

    \begin{frame}{Git model}
        \begin{center}
            \alert{D}irected \alert{A}cyclic \alert{G}raph
            %\only<1>{\alert{D}irected \alert{A}cyclic \alert{G}raph}
            %\only<2>{File system!}
        \end{center}
    \end{frame}

    \begin{frame}{What is Git?}
        \begin{center}
            \only<1>{\alert{Git} is a Persistent Map.}
            \only<2>{\alert{Git} is a Stupid Content Tracker.}
        \end{center}
    \end{frame}

    \begin{frame}{Branch}
        \begin{center}
            List local branches\\[1em]
            \texttt{git branch}
        \end{center}
        \note[item]{Katalog \texttt{refs/heads}}
    \end{frame}

    \begin{frame}{Branch}
        \begin{center}
            A \alert{branch} is just a reference to a commit.
        \end{center}
    \end{frame}

    \begin{frame}{Branch}
        \begin{center}
            Create a new \alert{branch}\\[1em]
            \texttt{git branch <branch name>}
        \end{center}
        \note[item]{Komenda \texttt{git branch second}}
        \note[item]{Komenda \texttt{git branch}}
        \note[item]{Jeden oznaczony gwiazdką. Co to znaczy?}
        \note[item]{\texttt{open .git} --- plik \texttt{HEAD}}
    \end{frame}

    \begin{frame}{Branch}
        \begin{center}
            \only<1>{\alert{HEAD} --- current branch}
            \only<2>{\alert{HEAD} is just a reference to a branch.}
        \end{center}
        \note[item]{Zmodyfikuj plik \texttt{config/new\_file.txt}}
        \note[item]{\texttt{git status}}
        \note[item]{\texttt{git commit -m "3rd commit"}}
        \note[item]{\texttt{git log}}
        \note[item]{Podsumowanie, na co wskazują branche i HEAD}
        \note[item]{\texttt{git switch second}}
        \note[item]{Wspomij o \texttt{git checkout}}
        \note[item]{Gdzie teraz wskazuje \texttt{HEAD}}
        \note[item]{Pokaż zawartość pliku \texttt{config/new\_file.txt}}
        \note[item]{Zmodyfikuj plik \texttt{config/new\_file.txt}}
        \note[item]{\texttt{git status}}
        \note[item]{\texttt{git commit -m "4th commit"}}
        \note[item]{\texttt{git log}}
        \note[item]{\texttt{git switch master}}
    \end{frame}

    \begin{frame}{Merge}
        \begin{center}
            Merging.
        \end{center}
        \note[item]{\texttt{git switch master}}
        \note[item]{\texttt{git merge second}}
        \note[item]{Konflikt!}
        \note[item]{\texttt{git status}}
        \note[item]{\texttt{vim config/new\_file.txt}}
        \note[item]{Rozwiąż konflikt}
        \note[item]{\texttt{git status}}
        \note[item]{\texttt{git add config/new\_file.txt}}
        \note[item]{\texttt{git status}}
        \note[item]{\texttt{git commit}}
        \note[item]{\texttt{git log}}
        \note[item]{\texttt{git cat-file -p} (top commit)}
        \note[item]{Dwóch przodków}
    \end{frame}

    \begin{frame}{References}
        \begin{center}
            \alert{References} between commits are used to track history.\\
            \pause
            All the other \alert{references} are used to track content.
        \end{center}
    \end{frame}

    \begin{frame}{References}
        \begin{center}
            When you move to another commit (switch to another branch), \alert{Git} doesn't care about history.\\[1em]
            \pause
            It just cares about trees and blobs.
        \end{center}
    \end{frame}

    \begin{frame}{Git model}
        \begin{center}
            \only<1>{\alert{Git's} content management is simpler than it looks.}
            \only<2>{\alert{Git} (mostly) doesn't care about your working directory.}
            \only<3>{Now we can forget about trees and blobs  \faSmileO.}
        \end{center}
    \end{frame}

    \begin{frame}{Merge}
        \begin{center}
            \only<1>{Merging without merging.}
            \only<2>{\alert{Fast-forward} merge.}
        \end{center}
        \note[item]{\texttt{git switch second}}
        \note[item]{\texttt{git merge master}}
        \note[item]{\texttt{Nie ma nowego merge commita}}
    \end{frame}

    \begin{frame}{Detached head}
        \begin{center}
            \only<1>{Checkout the commit.}
        \end{center}
        \note[item]{\texttt{git switch master}}
        \note[item]{\texttt{git checkout ...}}
        \note[item]{\texttt{cat .git/HEAD}}
        \note[item]{Pokazuje na commit, a nie na branch}
        \note[item]{\texttt{git branch}}
        \note[item]{Zmodyfikuj plik \texttt{config/new\_file.txt}}
        \note[item]{\texttt{git add config}}
        \note[item]{\texttt{git commit -m "5th commit"}}
        \note[item]{\texttt{git switch master}}
        \note[item]{\texttt{git log}}
        \note[item]{Gdzie ten commit?}
    \end{frame}

    \begin{frame}{Three rules}
        \begin{center}
            \onslide<1->{The current branch tracks new commits.}

            \onslide<2->{When you move to another commit,\\\alert{Git} updates your working directory.}

            \onslide<3->{Unraechable objects are garbage collected.}
        \end{center}
    \end{frame}

    \begin{frame}{Rebase}
        \begin{center}
            How \alert{rebase} works?
        \end{center}
        \note[item]{Przygotuj branch three i dwa nowe commity}
        \note[item]{Na branch master również przygotuj dwa commity}
        \note[item]{\texttt{git switch three}}
        \note[item]{\texttt{git rebase master}}
        \note[item]{\texttt{git switch master}}
        \note[item]{\texttt{git rebase three}}
    \end{frame}

    \begin{frame}{Rebase}
        \begin{center}
            \only<1>{Commits have different hashes. Why?}
            \only<2>{\alert{Git} garbage-collects unreachable objects.}
        \end{center}
    \end{frame}

    \begin{frame}{Merge vs Rebase}
        \begin{center}
            \only<1>{\alert{Merges} preserve history.}
            \only<2>{\alert{Rebases} refactor history.}
            \only<3>{When in doubt, just merge.}
        \end{center}
    \end{frame}

    \begin{frame}{Tags}
        \begin{center}
            \only<1>{A \alert{tag} is a label for a commit.}
        \end{center}
        \note[item]{\texttt{git tag v0.1.0}}
        \note[item]{\texttt{git tag v0.1.0 -a -m "New version"}}
        \note[item]{\texttt{git tag}}
        \note[item]{\texttt{git checkout v0.1.0}}
        \note[item]{\texttt{open .git/refs/tags}}
        \note[item]{\texttt{git cat-file -t ...}}
        \note[item]{\texttt{git cat-file -p ...}}
    \end{frame}

    \begin{frame}{Tags}
        \begin{center}
            \only<1>{\alert{Annotated} tags and \alert{"Lightweight"} tags.}
            \only<2>{A \alert{tag} is like a branch that doesn't move.}
        \end{center}
    \end{frame}

    \begin{frame}{What is Git?}
        \begin{center}
            \alert{Git} is a Version Control System.
        \end{center}
    \end{frame}

    \begin{frame}{Repository}
        \begin{center}
            Multiple repos.
        \end{center}
        \note[item]{Repo on GitHub}
        \note[item]{\texttt{git clone ...}}
        \note[item]{Tylko jeden lokalny branch}
    \end{frame}

    \begin{frame}{Repository}
        \begin{center}
            We have two clones of the repository.
        \end{center}
    \end{frame}

    \begin{frame}{Repository}
        \begin{center}
            \alert{Local} and \alert{remote} repositories.
        \end{center}
        \note[item]{\texttt{vim .git/config}}
        \note[item]{Ważna sekcja \texttt{remote}}
        \note[item]{Domyślny remote: \texttt{origin}}
        \note[item]{\texttt{git branch}}
        \note[item]{\texttt{git branch --all}}
        \note[item]{\texttt{open .git/refs/remotes}}
        \note[item]{Jest \texttt{HEAD}, a nie ma branchy}
        \note[item]{\texttt{packed-refs}}
    \end{frame}

    \begin{frame}{Repository}
        \begin{center}
            Like a local branch, a \alert{remote branch} is just a reference to a commit.
        \end{center}
        \note[item]{\texttt{git show-ref master}}
        \note[item]{\texttt{git show-ref second}}
    \end{frame}

    \begin{frame}{Repository}
        \begin{center}
            Synchronizing repositories.
        \end{center}
    \end{frame}

    \begin{frame}{Values and hashes}
        \begin{center}
            \begin{tabular}{@{}l@{}@{}l l@{}}
                value: & & \texttt{test} \\
                key:   & & \texttt{9daeafb9864cf43055ae93beb0afd6c7d144bfa4}
            \end{tabular}
        \end{center}
        \note[item]{Obiekty są niezmienne i unikalne.}
    \end{frame}

    \begin{frame}{Synchronizing repositories}
        \begin{center}
            Push commits.
        \end{center}
        \note[item]{Najpierw git pobrał wszystkie obiekty z repozytorium zdalnego.}
        \note[item]{Teraz git musi wysłać nowe obiekty.}
        \note[item]{Zmodyfikuj \texttt{main\_file.txt}}
        \note[item]{\texttt{git status}}
        \note[item]{\texttt{git add main\_file.txt}}
        \note[item]{\texttt{git commit -m "4th commit"}}
        \note[item]{\texttt{git show-ref master}}
        \note[item]{\texttt{git push}}
        \note[item]{\texttt{git show-ref master}}
        \note[item]{Czasem potrzebny najpierw \texttt{git fetch} i \texttt{git merge}, czyli razem \texttt{git pull}}
    \end{frame}

    \begin{frame}{Synchronizing repositories}
        \begin{center}
            \only<1>{You \alert{fetch} and \alert{merge}, then you \alert{push}.}
            \only<2>{You \alert{pull}, then you \alert{push}.}
        \end{center}
    \end{frame}

    \begin{frame}{Synchronizing repositories}
        \begin{center}
            \only<1>{Rebase revisited.}
            \only<2>{Never rebase shared commits.}
        \end{center}
    \end{frame}

    \begin{frame}{What is Git?}
        \begin{center}
            \alert{Git} is a Distributed Version Control System.
        \end{center}
    \end{frame}

    \begin{frame}[standout]
        Thank you! \faSmileO
    \end{frame}

    \appendix

    \begin{frame}{Z shell (Zsh)}
        \begin{itemize}
            \item\href{https://en.wikipedia.org/wiki/Z\_shell}{Z shell}
            \begin{itemize}
                \item\href{https://github.com/ohmyzsh/ohmyzsh/wiki/Installing-ZSH}{Installing}
            \end{itemize}
            \item\href{https://ohmyz.sh}{Oh My Zsh}
            \begin{itemize}
                \item\href{https://ohmyz.sh/\#install}{Installing}
                \item\href{https://github.com/ohmyzsh/ohmyzsh/wiki/Themes}{Themes} (\href{https://github.com/romkatv/powerlevel10k\#installation}{Powerlevel10k})
                \item Plugins
                \begin{itemize}
                    \item\href{https://github.com/zsh-users/zsh-syntax-highlighting/blob/master/INSTALL.md}{Syntax highlighting}
                    \item\href{https://github.com/zsh-users/zsh-autosuggestions/blob/master/INSTALL.md\#oh-my-zsh}{Autosuggestions}
                \end{itemize}
            \end{itemize}
        \end{itemize}
    \end{frame}

    \begin{frame}[fragile]{My Git config}
        \begin{verbatim}

    user.name=Jaroslaw Karczmarczyk
    user.email=jaroslaw.karczmarczyk@pwc.com
    help.autocorrect=1
    diff.algorithm=histogram
    core.pager=less -RFX
    core.editor=vim
        \end{verbatim}
    \end{frame}

    \begin{frame}[fragile]{My Git config}
        \begin{verbatim}

    core.abbrev=7
    core.autocrlf=input
    core.filemode=true
    pull.rebase=true
    merge.conflictstyle=diff3
    color.ui=auto
        \end{verbatim}
    \end{frame}
\end{document}
