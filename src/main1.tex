\documentclass{beamer}
\usepackage{graphicx}
\usepackage{hyperref}
\usepackage{appendixnumberbeamer}

\usetheme[numbering=fraction,block=fill]{metropolis}
\usecolortheme{owl}

\hypersetup{
    colorlinks=false,
    pdftitle={Introduction to Git},
    pdfpagemode=FullScreen,
    pdfauthor={Jaroslaw Karczmarczyk}
}

\urlstyle{same}
\graphicspath{ {../images/} }

\setbeameroption{show notes on second screen=right}
\setbeamertemplate{note page}[plain]
\setbeamerfont{note page}{size=\tiny}
\addtobeamertemplate{note page}{\setbeamerfont{itemize/enumerate subbody}{size=\tiny}}{}

\title{Introduction to \includegraphics[height=1.2ex]{git-logo-white}}
\date{\today}
\author[Jarek]{Jarosław Karczmarczyk\\\texttt{jaroslaw.karczmarczyk@pwc.com}}
\institute{PwC Advisory spółka z ograniczoną odpowiedzialnością sp.k.\\Data \& Analytics}
\logo{\includegraphics[scale=0.08]{pwc-logo-white}}
\titlegraphic{\hfill\includegraphics[scale=0.12]{pwc-logo-white}}
\setbeamercolor{alerted text}{fg=OwlYellow}
\setbeamercolor*{block title}{
    use=normal text,
    fg=normal text.fg,
    bg=normal text.bg!80!fg
}
\setbeamercolor{block body}{
    use={block title, normal text},
    bg=block title.bg!50!normal text.bg
}

\setbeamertemplate{itemize items}[triangle]
\begin{document}
    \maketitle


%    \begin{frame}{Introduction}
%        \begin{block}{What is Git?}
%            \alert{Git} is an Open Source Distributed Version Control System.
%        \end{block}
%        \pause
%        That’s a lot of words to define Git.
%    \end{frame}
%
%    \begin{frame}{Introduction}
%        \begin{block}{Control System}
%            This basically means that \alert{Git} is a \alert{content tracker}.
%            So \alert{Git} can be used to store content --- it is mostly used to store code due to the other features it provides.
%        \end{block}
%    \end{frame}
%
%    \begin{frame}{Introduction}
%        \begin{block}{Version Control System}
%            The code which is stored in \alert{Git} keeps changing as more code is added.
%            Also, many developers can add code in parallel.
%            So \alert{Version Control System} helps in handling this by maintaining a history of what changes have happened.
%            Also, \alert{Git} provides features like branches and merges.
%        \end{block}
%    \end{frame}
%
%    \begin{frame}{Introduction}
%        \begin{block}{Distributed Version Control System}
%            \alert{Git} has a remote repository which is stored in a server and a local repository which is stored in the computer of each developer.
%            This means that the code is not just stored in a central server, but the full copy of the code is present in all the developers’ computers. \alert{Git} is a \alert{Distributed Version Control System} since the code is present in every developer’s computer.
%        \end{block}
%    \end{frame}

    \begin{frame}{Introduction}
        Porcalain commands --- high level commands
        \begin{itemize}
            \item\texttt{git status}
            \item\texttt{git add}
            \item\texttt{git commit}
            \item\texttt{git push}
            \item\texttt{git pull}
            \item\texttt{git branch}
            \item\texttt{git switch} (\texttt{git checkout})
            \item\texttt{git merge}
            \item\texttt{git rebase}
            \item\texttt{git pull}
            \item\texttt{git push}
            \item\texttt{...}
        \end{itemize}
    \end{frame}

    \begin{frame}{Introduction}
        Plumbing commands --- low level commands
        \begin{itemize}
            \item\texttt{git cat-file}
            \item\texttt{git hash-object}
            \item\texttt{git count-objects}
            \item\texttt{...}
        \end{itemize}
    \end{frame}

    \begin{frame}{Introduction}
        \begin{center}
            If you want to master \alert{Git}, don't worry\\about learning the commands.
        \end{center}
        \pause
        \begin{center}
            Instead, learn the model.
        \end{center}
    \end{frame}

    \begin{frame}{What is Git?}
        \begin{center}
            \alert{Git} is a Distributed Version Control System.
        \end{center}
        \pause
        \begin{center}
            That’s a lot of words to define Git.
        \end{center}
    \end{frame}

    \begin{frame}{What is Git?}
        \begin{center}
            \only<1>{\alert{Git} is a Version Control System.}
            \only<2>{\alert{Git} is a Stupid Content Tracker.}
            \only<3>{\alert{Git} is a Persistent Map.}
        \end{center}
    \end{frame}

    \begin{frame}{Values and keys}
        \begin{center}
            \only<1>{Value is a sequence on bytes.}
            \only<2>{Key is a SHA1\footnote{\href{https://en.wikipedia.org/wiki/SHA-1}{https://en.wikipedia.org/wiki/SHA-1}} hash.}
        \end{center}
    \end{frame}

    \begin{frame}{Values and keys}
        \begin{center}
            \begin{tabular}{ l l }
                value:& test\\
                key:& \texttt{9daeafb9864cf43055ae93beb0afd6c7d144bfa4}\\
                \\
            \end{tabular}
            \pause
            \texttt{echo "test" | git hash-object --stdin}
        \end{center}
    \end{frame}

    \begin{frame}{Values and keys}
        \begin{center}
            Every object in \alert{Git} has its own SHA1 hash.
        \end{center}
        \pause
        \begin{center}
            So, what if they collide?
        \end{center}
    \end{frame}

    \begin{frame}{A persistent map}
        \begin{center}
            \only<1>{\texttt{echo "test" | git hash-object --stdin -w}}
            \only<2>{\texttt{git cat-file -t 9daeafb9864cf43055ae93beb0afd6c7d144bfa4} (type)}
            \only<3>{\texttt{git cat-file -p 9daeafb9864cf43055ae93beb0afd6c7d144bfa4} (pretty print)}
        \end{center}
    \end{frame}

    \begin{frame}{Example}
        \begin{center}
            First commit!
        \end{center}
        \note[item]{Przejdz do katalogu \texttt{\textasciitilde/projects/training}}
        \note[item]{\texttt{tree .}}
        \note[item]{\texttt{git init}}
        \note[item]{\texttt{l} --- katolog \texttt{.git}}
        \note[item]{katalog \texttt{.git/objects} jest pusty}
        \note[item]{\texttt{git status}}
        \note[item]{\texttt{git add main\_file.txt}}
        \note[item]{\texttt{git add config}}
        \note[item]{\texttt{git status}}
        \note[item]{\texttt{git commit}}
        \note[item]{\texttt{git status}}
        \note[item]{\texttt{git log}}
        \note[item]{\texttt{open .git}}
        \note[item]{Pokaż za pomoca \texttt{git cat-file -p} zawartość commita, tree i blobs}
        \note[item]{Zmodyfikuj \texttt{main\_file.txt}}
        \note[item]{\texttt{git status}}
        \note[item]{\texttt{git add main\_file.txt}}
        \note[item]{\texttt{git commit -m "Second commit"}}
        \note[item]{\texttt{git status}}
        \note[item]{\texttt{git log}}
        \note[item]{Pokaż za pomoca \texttt{git cat-file -p} zawartość commita --- parent}
        \note[item]{Pożaż, jak zmieniło się tree}
    \end{frame}

    \begin{frame}{Git objects}
        \begin{itemize}[<+-| alert@+>]
            \item Blobs
            \item Trees
            \item Commits
            \item Annotated Tags
        \end{itemize}
    \end{frame}

    \begin{frame}{Git model}
        \begin{center}
            \only<1>{\alert{D}irected \alert{A}cyclic \alert{G}raph}
            \only<2>{File system!}
        \end{center}
    \end{frame}

    \begin{frame}{What is Git?}
        \begin{center}
            \only<1>{\alert{Git} is a Persistent Map.}
            \only<2>{\alert{Git} is a Stupid Content Tracker.}
        \end{center}
    \end{frame}

    \setbeamercovered{transparent}

    \begin{frame}{Branch}
        \begin{center}
            List local branches\\[1em]
            \texttt{git branch}
        \end{center}
    \end{frame}

    \begin{frame}{Branch}
        \begin{center}
            A branch is just a reference to a commit.
        \end{center}
    \end{frame}

    \begin{frame}{Branch}
        \begin{center}
            Create a new branch\\[1em]
            \texttt{git branch \textsl{<branch name>}}
        \end{center}
    \end{frame}

    \begin{frame}{Branch}
        \begin{center}
            \only<1>{\alert{HEAD} --- current branch}
            \only<2>{\alert{HEAD} is just a reference to a branch}
        \end{center}
    \end{frame}

    \begin{frame}{Git areas}
        \begin{itemize}[<+-| alert@+>]
            \item Working Area
            \item Index
            \begin{itemize}
                \item<2-> \texttt{git diff}
                \item<2-> \texttt{git diff --cached}
            \end{itemize}
            \item Repository
            \item Stash
        \end{itemize}
    \end{frame}

    \begin{frame}{Two fundametal questions about commands}
        \begin{itemize}[<+->]
            \item Jak komenda przesuwa informacje w 4 przestrzeniach?
            \item Jak komenda zmienia repozytorium?
        \end{itemize}
    \end{frame}

    \begin{frame}{Example}
        \begin{itemize}[<+-| alert@+>]
            \item Zmiana w pliku (WA)
            \item git add (index) (-p) (-i)
            \item git commit (repo)
        \end{itemize}
    \end{frame}

    \begin{frame}{Checkout}
        \begin{enumerate}[<+-| alert@+>]
            \item git checkout
            \item przesuwa head i branch
            \item kopiuje dane z repo do indexu i wa
        \end{enumerate}
    \end{frame}

    \begin{frame}{Remove file}
        \begin{center}
            \texttt{git rm <path>}
        \end{center}
    \end{frame}

    \begin{frame}{Rename file}
        \begin{center}
            TODO
        \end{center}
    \end{frame}

    \begin{frame}{Reset}
        trzeba wiedziec o 3 areas i jak dzialaja branche

        reset robi rozne rzeczy w roznych kontekstach

        komendy ktore poruszaja branche: commit merge rebase pull
        \begin{enumerate}[<+-| alert@+>]
            \item\texttt{git reset --soft} --- soft przesuwa branch
            \item\texttt{git reset --mixed} --- przesuwa branch, kopiuje dane do indexu
            \item\texttt{git reset --hard} --- przesuwa branch, kopiuje dane do indexu i wa
        \end{enumerate}
    \end{frame}

    \begin{frame}{Stash}
        \begin{itemize}[<+-| alert@+>]
            \item\texttt{git stash save}
            \item\texttt{git stash}
            \item\texttt{git stash --keep-index --include-untracked (-ku)}
            \item\texttt{git stash apply}
            \item\texttt{git stash clear}
            \item\texttt{git stash pop}
            \item\texttt{git stash list}
        \end{itemize}
    \end{frame}

    \begin{frame}{Log}
        \begin{itemize}[<+-| alert@+>]
            \item\texttt{git log}
            \item\texttt{git log --graph}
            \item\texttt{git log --decoreate}
            \item\texttt{git log --oneline}
        \end{itemize}
    \end{frame}

    \begin{frame}{Show}
        \begin{itemize}[<+-| alert@+>]
            \item\texttt{git show} (hash,HEAD,HEAD\textasciicircum --- parent commit,branch)
            \item\texttt{git show HEAD~2\textasciicircum2} --- second parent
            \item\texttt{git show HEAD@\{"1 month ago"\}} --- second parent
            \item\texttt{git show --no-patch}
            \item\texttt{git show --stat}
        \end{itemize}
    \end{frame}

    \begin{frame}{Blame}
        \begin{center}
            \texttt{git blame}
        \end{center}
    \end{frame}

    \begin{frame}{Show}
        \begin{itemize}[<+-| alert@+>]
            \item\texttt{git diff HEAD HEAD\textasciitilde2}
            \item\texttt{git diff master new}
            \item\texttt{git diff --word-diff}
            \item\texttt{--unified=10 (3)}
        \end{itemize}
    \end{frame}

    \begin{frame}{Log}
        \begin{center}
            \begin{itemize}[<+-| alert@+>]
                \item\texttt{git log}
                \item\texttt{git log --follow <filename>}
                \item\texttt{git log --grep apples --oneline}
                \item\texttt{git log -Gapples -p}  (dadanie albo usuniecie albo modyfikacja) --- regex
                \item\texttt{git log -Sapples} (dadanie albo usuniecie) --- string
                \item\texttt{git log -3 --oneline}
                \item\texttt{git log HEAD\textasciitilde5..HEAD\textasciicircum{} --oneline}
                \item\texttt{git log master..new --oneline}
            \end{itemize}
        \end{center}
    \end{frame}

    \begin{frame}{Log}
        \begin{center}
            \texttt{git log --format="\%h | \%d \%s (\%cr) [\%an]"}
        \end{center}

        \pause
        \texttt{\%h} --- abbreviated commit hash

        \texttt{\%d} --- ref names

        \texttt{\%s} --- subject

        \texttt{\%b} --- body

        \texttt{\%cr} --- committer date, relative

        \texttt{\%an} --- author name
    \end{frame}

    \begin{frame}{Log format}
        \only<1>{\texttt{git log --format="\%Cred\%h\%Creset | \%C(yellow)\%d\%Creset \%s \%Cgreen(\%cr)\%Creset \%C(cyan)[\%an]\%Creset"}}

        \only<2>{\texttt{git log --format="\%Cred\%h\%Creset | \%C(yellow)\%d\%Creset \%s \%Cgreen(\%cr)\%Creset \%C(cyan)[\%an]\%Creset" --graph --all}}
    \end{frame}

    \begin{frame}{Commit}
        \texttt{git commit --amend (-C HEAD)}
    \end{frame}

    \begin{frame}{Commit}
        \texttt{git rebase -i HEAD\textasciitilde5}
    \end{frame}

    \begin{frame}{Branch}
        \begin{itemize}[<+-| alert@+>]
            \item\texttt{git branch}
            \item\texttt{git branch <name>}
            \item\texttt{git branch -a}
            \item\texttt{git branch -d <name>}
            \item\texttt{git branch -D <name>}
        \end{itemize}
    \end{frame}

    \begin{frame}{Reflog}
        \begin{itemize}[<+-| alert@+>]
            \item\texttt{git reflog}
            \item\texttt{git reflog refs/heads/master}
        \end{itemize}
    \end{frame}

    \begin{frame}{Revert commit}
        \texttt{git revert <hash>} (uwaga na revert merge commit)
    \end{frame}

    \begin{frame}{Config}
        \begin{itemize}[<+-| alert@+>]
            \item\texttt{git config}
            \item\texttt{git config --global alias.st status}
        \end{itemize}
    \end{frame}

    \begin{frame}{Config}
        \begin{itemize}[<+-| alert@+>]
            \item\texttt{git remote}
            \item\texttt{git remote -v}
            \item\texttt{git remote add <name> <url>}
            \item\texttt{git remote rename <old> <new>}
            \item\texttt{git remote rm <name>}
            \item\texttt{git remote show <name>}
        \end{itemize}
    \end{frame}

    \begin{frame}{Pull}
        \begin{center}
            \texttt{git pull <name>}
        \end{center}
    \end{frame}

    \begin{frame}{Push}
        \begin{center}
            \texttt{git push}
        \end{center}
    \end{frame}

    \begin{frame}{Merge}
        \begin{center}
            \texttt{git merge} (conflict git add; git commit)
        \end{center}
    \end{frame}

    \begin{frame}{Push}
        \begin{center}
            \texttt{git show-ref master}
        \end{center}
    \end{frame}

    \begin{frame}[standout]
        Thank you!
    \end{frame}

    \appendix

    \begin{frame}{Z shell (Zsh)}
        \begin{itemize}
            \item\href{https://en.wikipedia.org/wiki/Z_shell}{Z shell}
            \begin{itemize}
                \item\href{https://github.com/ohmyzsh/ohmyzsh/wiki/Installing-ZSH}{Installing}
            \end{itemize}
            \item\href{https://ohmyz.sh/}{Oh My Zsh}
            \begin{itemize}
                \item\href{https://ohmyz.sh/\#install}{Installing}
                \item\href{https://github.com/ohmyzsh/ohmyzsh/wiki/Themes}{Themes}
                \item Plugins
                \begin{itemize}
                    \item\href{https://github.com/zsh-users/zsh-syntax-highlighting/blob/master/INSTALL.md}{Syntax highlighting}
                    \item\href{https://github.com/zsh-users/zsh-autosuggestions/blob/master/INSTALL.md\#oh-my-zsh}{Autosuggestions}
                \end{itemize}
            \end{itemize}
        \end{itemize}
    \end{frame}
\end{document}
